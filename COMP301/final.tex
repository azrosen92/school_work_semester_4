\documentclass[12pt]{article}

\usepackage{amssymb}
\usepackage{amsmath}
\usepackage{nicefrac}

\begin{document}
\title{Final Exam}
\author{Aaron Rosen}
\date{Monday, May 13}
\maketitle

\section*{Problem 1}
$\mathbb{F} = \{0^x : x$ is the $n^{th}$ fibonacci number with $n \geq 0 \}$\\
\\
Consider the following strings: \\

$0^1, 0^2, 0^3 \in \mathbb{F}$\\
\\
Concatenating $0$ we get: \\

$0^10 = 0^2 \in \mathbb{F}$

$0^20 = 0^3 \in \mathbb{F}$

$0^30 = 0^4 \notin \mathbb{F}$\\
\\
Now for any $k$ and any $j > 0$\\

$0^k 0^j \notin \mathbb{F}$ unless $j=k-1$\\
\\
Therefore\\

$0^{k+j} \not \equiv _{R} 0^k 0^j$ for all $k$ and $j$\\
\\
So all members of $\mathbb{F}$ except $0^1$ and $0^2$ are pairwise non-equivalent,
hence\\

$\lvert \nicefrac{\Sigma}{\not \equiv _R} \rvert = \infty$\\
\\
So, by the Myhill-Nerode Theorem, $\mathbb{F}$ is not regular.


\section*{Problem 2}
We want to show an r.e. set is recursive iff an enumeration machine enumerates 
it in increasing order.\\
\\
$\Leftarrow$ Say $A$ is enumerated by an enumeration machine $E$ in increasing
order. Let a TM $M$, accept $A$. On input $x$, $A$ is enumerated by $E$ and $M$
waits to see if $x$ comes up. If it does, $M$ halts and enters an accept state. If
a number greater then $x$ comes up, $M$ enters a reject state and halts. Since $A$
is enumerated in increasing order, $M$ will always either halt and accept or halt
and reject so it is recursive.\\
\\
$\Rightarrow$ Suppose $A$ is recursive, that is, there is a total TM $M$ such that 
$A = L(M)$. Since $A$ is recursive, $M$ either accepts or rejects, but it always
halts. We define $M$ as always accepting as long as it is given numbers in
increasing order. Therefore the set enumerated in increasing order will be 
accepted.\\


\section*{Problem 3}
We want to show $A$ is r.e. iff there is a recursive binary relation $R$ such that 
$A = \{x : \exists y (R(x,y))\}$\\
\\
$\Rightarrow$ Assume $A$ is r.e. This means there is some machine, we will call it
$M_A$ such that $A = L(M_A)$. First we will define a binary relation $R$ as follows:\\

$(x,y) \in R \iff M_A(x) \downarrow a$ in $y$ steps $\iff M_A^y(x) \downarrow a$\\
\\
Now with this $R$ we get\\

$A = \{x : \exists y (M^y_A(x) \downarrow a)\}$\\
\\
Which means $A$ consists of all strings $x$ accepted by $M_A$ in a finite number of
steps. Since $A$ is r.e.\ this is exactly what we want. And there is recursive
binary relation $R$ such that $A = \{x : \exists y (R(x,y))\}$\\
\\
$\Leftarrow$ Assume there is a recursive binary relation $R$ such that 
$A = \{x : \exists y (R(x,y)) \}$. We know the set $\{x\#y : R(x,y)\}$ is recursive.
Then there is a TM $M_R$ which, given $x\#y$ either accepts or rejects\\

$M_R(x\#y) = \begin{cases} 
                M_R(x\#y) \downarrow a, & \mbox{if } (x,y) \in R\\
                M_R(x\#y) \downarrow r, & \mbox{if } (x,y) \notin R
            \end{cases}$\\
\\
Now we will build a TM $M_A$ that simulates $M_R$ such that\\

$(x,y) \in R \Rightarrow M_R(x\#y) \downarrow a \Rightarrow M_A(x) \downarrow a$

$(x,y) \notin R \Rightarrow M_R(x\#y) \downarrow r \Rightarrow M_A(x) \mbox{ does not accept}$\\
\\
So given any $x$ we can try all $y$'s until we find one such that\\

$M_R(x\#y)$ accepts\\
\\
and if so $M_A(x)$ accepts. Otherwise, if we never find a $y$ such that $M_R$ 
accepts, then $M_A(x)$ diverges. Hence $A$ is r.e.


\section*{Problem 4}
$TOT = \{M : M \mbox{ is a total TM} \}$\\
\\
First we will show that $\overline{TOT}$ is not r.e. using the reduction\\

$\overline{HP} \leqslant _m ^{\sigma} \overline{TOT}$\\
\\
We know $\overline{HP}$ is not r.e.\ because $HP$ is r.e.\ and not recursive. Now we
need a $\sigma$ such that\\

$M\#x \in \overline{HP} \Rightarrow \sigma(M\#x) \in \overline{TOT}$

$M\#x \notin \overline{HP} \Rightarrow \sigma(M\#x) \notin \overline{TOT}$\\
\\
Say we use $\sigma(M\#x)$ runs $M$ on input $x$\\
\begin{align*}
  M\#x \in \overline{HP} & \Rightarrow M(x)\uparrow\\
                          & \Rightarrow \sigma(M\#x) \mbox{ diverges on all inputs}\\
                          & \Rightarrow \sigma(M\#x) \in \overline{TOT}\\
  M\#x \notin \overline{HP} & \Rightarrow M(x) \downarrow\\ 
                            & \Rightarrow \sigma(M\#x) \mbox{ halts on all inputs}\\
                            & \Rightarrow \sigma(M\#x) \notin \overline{TOT}
\end{align*}
Therefore $\overline{HP}$ reduces to $\overline{TOT}$ and $\overline{TOT}$ is
not r.e.\\
\\
Now we will show $TOT$ is not r.e.\\
Assume, for the sake of contradiction, that $TOT$ is r.e. Then there exists a TM $K$
such that\\

$L(K) = TOT$\\
\\
Thus on input $M\#x$\\

$K(M\#x) = \begin{cases}
              \downarrow a, & \mbox{if } M(x) \mbox{ halts}\\
              \downarrow r \mbox{ or} \uparrow, & \mbox{if } M(x) \mbox{ diverges}
            \end{cases}$\\
\\
Now consider a TM $N$ defined as\\

$N(x) = \begin{cases}
          \downarrow a, & \mbox{if } K(M_x\#x) \mbox{ does not accept}\\
          \downarrow r, & \mbox{if } K(M_x\#x) \downarrow a
        \end{cases}$\\
\\            
So for any $x$
\begin{align*}
  N(x) \downarrow a & \mbox{ iff } K(M_x\#x) \mbox{ does not accept} \\
                    & \mbox{ iff } M_x(x) \mbox{ diverges}
\end{align*}
Therefore $N(x)$ was not enumerated by E, and we've reached a contradiction. So 
$TOT$ is not r.e.

\section*{Problem 5}
$L = \{M : M$ accepts at least $n$ strings $\}$\\
\\
First we will show $L$ is r.e.\ by building a TM $M_L$ that accepts it. $M_L$ takes
as input a machine, we'll call it $Q$, and a list of binary strings, delimited by 
the dollar sign $\$$ in increasing order of size (i.e.\ 
$\$0\$1\$00\$01\$11\$000\$...$) for each string $x$ in the list of binary strings
$Q$ will run for $|x|$ steps on $x$ and every string before it. It will continue
to do this until $Q$ accepts at least $n$ strings at which point $M_L$ will accept
it. If $Q$ never accepts $n$ strings then $M_L$ will loop. Therefore, we have
shown that $L$ is r.e.\\
\\
Now we will show that $\overline{L}$ is not r.e.\ by reducing a non-r.e.\ set to it.
We can use the reduction\\

$\overline{MP} \leqslant _m ^{\sigma} \overline{L}$\\
\\
Let $\sigma(M\#x)(y)$ ignore the input $y$ and run $M$ on $x$. If $M(x)$ accepts
then $\sigma(M\#x)(y)$ accepts, and if $M(x)$ does not accept then $\sigma(M\#x)(y)$
does not accept.
\begin{align*}
  M\#x \in \overline{MP} & \Rightarrow M(x) \mbox{ does not accept}\\ 
                         & \Rightarrow L(\sigma(M\#x)) = \emptyset\\
                         & \Rightarrow L(\sigma(M\#x)) \mbox{ has fewer than $n$ elements}\\
                         & \Rightarrow \sigma(M\#x) \in \overline{L}\\
  M\#x \notin \overline{MP} & \Rightarrow M(x) \downarrow a\\
                            & \Rightarrow L(\sigma(M\#x)) = \Sigma^*\\
                            & \Rightarrow L(\sigma(M\#x)) \mbox{ has more than $n$ elements}\\
                            & \Rightarrow \sigma(M\#x) \notin \overline{L}
\end{align*}
And since $\overline{MP}$ is not r.e.\ $\overline{L}$ must not be r.e.

\section*{*Problem 6}
Let $A$ and $B$ be r.e.\ and
$A / B = \{x : \exists y \in B (xy \in A) \}$. There are machines $M_A$ and $M_B$
such that $A = L(M_A)$ and $B = L(M_B)$\\
\\
Now we will design a machine $M$ that will accept the set $A / B$. $M$ simulates
the instructions of $M_A$ and $M_B$. Given an $x$, $M$ will dovetail the operation
of finding a $y$ in $B$ such that $xy \in A$. Since $B$ is r.e.\ there is an
enumeration machine $E_B$ that enumerates all of the elements of $B$. Going
through all of the elements of this enumeration, for the $n^{th}$ element in
$Enum(E_B)$, run $M_A$ for $n$ steps on $xy_i$ ($0 < i \leq n$).\\

$M(x)$ accepts if $M_A^n(xy_i)$ accepts for one of these $y_i$s

$M(x)$ rejects otherwise\\
\\
Therefore $A / B$ is r.e. 

\section*{*Problem 7}
\subsection*{(1)}
Assume, for the sake of contradiction $L(M) = L(N)$ is decidable. Then there is a 
total machine $Q$ such that\\

$Q(M\#N) = \begin{cases}
            \downarrow a, & \mbox{if } L(M) = L(N)\\
            \downarrow r, & \mbox{if } L(M) \neq L(N)
          \end{cases}$\\
\\
Let $L(N) = \{x\}$, then\\

$Q(M\#N) \downarrow a \Rightarrow L(M) = \{x\} \Rightarrow x \in L(M) \Rightarrow M\#x \in MP$\\
\\
But MP is undecidable, so $L(M) = L(N)$ is undecidable.
  
\subsection*{(2)}
Assume, for the sake of contradiction $L(M) \subseteq L(N)$ is decidable. Then there
is a total machine $Q$ such that\\

$Q(M\#N) = \begin{cases}
              \downarrow a, & \mbox{if } x \in L(M) \Rightarrow x \in L(N)\\
              \downarrow r, & \mbox{otherwise}
           \end{cases}$\\
\\
Let $L(M) = {x}$, then\\

    $Q(M\#N) \downarrow a \Rightarrow x \in L(N) \Rightarrow N\#x \in MP$\\
\\
But MP is undecidable, so $L(M) \subseteq L(N)$ is undecidable.

\subsection*{(3)}
Assume, for the sake of contradiction $L(M) \cap L(N) = \emptyset$ is decidable. Then 
there is a total machine $Q$ such that\\

$Q(M\#N) = \begin{cases}
              \downarrow a, & \mbox{if } x \in L(M) \Rightarrow x \notin L(N)\\
              \downarrow r, & \mbox{otherwise}
           \end{cases}$\\
\\
Let $L(N) = {x}$, then\\

$Q(M\#N) \downarrow a \Rightarrow x \notin L(M) \Rightarrow M\#x \notin MP$\\
\\
But MP is undecidable, so $L(M) \cap L(N) = \emptyset$ is undeciable.

\end{document}
